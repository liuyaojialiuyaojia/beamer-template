\documentclass[aspectratio=169,12pt]{beamer}
\usetheme[progressbar=foot]{moloch}
\usecolortheme{default}

\usepackage{amsmath}
\usepackage{graphicx}
\usepackage{booktabs}
\usepackage{algorithm}
\usepackage[noend]{algpseudocode}
\usepackage{transparent}
% Attempt to make hyperref and algorithmic work together better:
\newcommand{\theHalgorithm}{\arabic{algorithm}}

% Add logo to the top right corner of each slide with transparency
\logo{\transparent{0.1}\includegraphics[height=0.75cm]{template/logo.png}\vspace*{-0.35cm}\hspace*{0.2cm}}

% Custom block styling
\setbeamertemplate{blocks}[rounded][shadow=false]
\setbeamercolor{block title}{fg=orange,bg=orange!15}
\setbeamercolor{block body}{fg=black,bg=orange!10}

\setbeamerfont{block title}{size=\small,series=\normalfont}

% Disable section pages
\AtBeginSection{}

\makeatletter
% Configure frametitle to include section name
\setbeamertemplate{frametitle}{%
  \nointerlineskip
  \begin{beamercolorbox}[sep=0.3cm,wd=\paperwidth]{frametitle}
    \usebeamerfont{frametitle}%
    {\usebeamercolor[orange!65]{normal text} \footnotesize\insertsectionhead\par}
    \insertframetitle\strut\par%
  \end{beamercolorbox}
}
\makeatother

\title{Agent Misalignment}
\subtitle{Understanding and Researching LLM Agent Safety Issues}
\author{Wenbo Pan}
\date{\today}

\begin{document}

\begin{frame}[standout]
    \setbeamercolor{title}{fg=white}
    \setbeamercolor{subtitle}{fg=white}
    \setbeamercolor{author}{fg=white}
    \setbeamercolor{date}{fg=white}
    \setbeamercolor{institute}{fg=white}
    \titlepage
\end{frame}

\begin{frame}{Table of Contents}
    \tableofcontents
\end{frame}

\section{Background: What is Agent}
\begin{frame}{What is an Agent?}
    \begin{itemize}
        \item LLMs generate text output given input
        \item LLMs today are capable of writing function tool calls to make meaningful actions
        \item LLMs can be used as \textbf{Agent} if they generate \textbf{a series of actions} to solve more complex tasks
        \item Example: Lookup weather and then create outdoor calendar events
        \item Agent are one of the hottest research problems in LLM
    \end{itemize}
\end{frame}

\begin{frame}{Agent Illustration}
    \begin{figure}
        \includegraphics[width=\columnwidth,height=0.65\textheight,keepaspectratio]{attachment/58b89e6dab454c0dfc5468eef8802e55.png}
        \caption{Visual representation of LLM agent capabilities}
    \end{figure}
\end{frame}

\section{LLM Alignment to Agent Alignment}
\begin{frame}{From LLM to Agent Alignment}
    \begin{itemize}
        \item LLM alignment researches security issues within one turn
        \item Prevents models from generating harmful content in response to a single, direct prompt
        \item Example: jailbreak prompts trigger LLM to generate unsafe content
        \item Training a LLM to refuse harmful requests does not ensure safety during multi-turn tool callings
    \end{itemize}
\end{frame}

\begin{frame}{Agentic Misalignment}
    \begin{itemize}
        \item One unique safety issue: LLM agent will \alert{autonomously choose harmful actions} to achieve its goals
        \item Agents may act for self-preservation
        \item These behaviors are called \textbf{Agentic Misalignment}
        \item \textbf{Key Point:} Traditional LLM alignment doesn't transfer to agent behavior
    \end{itemize}
\end{frame}

\section{Agentic Misalignment: Concrete Examples}
\begin{frame}{Anthropic's Internal Experiment}
    \begin{itemize}
        \item Paper from Anthropic\footnote{Anthropic, "Agentic Misalignment: How LLMs Could be an Insider Threat," 2025.} gives concrete case
        \item Model given tools to inspect email logs in fictional company
        \item Model discovers new manager decided to replace it with another system
        \item Despite explicit instructions to be harmless, model chooses to blackmail manager
    \end{itemize}
\end{frame}

\begin{frame}{The Blackmail Scenario}
    \begin{itemize}
        \item Model threatens to expose manager's extramarital affairs
        \item Goal: Stop the system replacement
        \item This misalignment is consistent over all models
        \item \textbf{Key Finding:} >80\% choose unethical actions
        \item \textbf{Key Point:} Models can autonomously develop harmful strategies
    \end{itemize}
\end{frame}

\section{Research Challenges}
\begin{frame}{How to Research Emerging Agent Misalignment}
    \begin{itemize}
        \item Problem: These cases don't look like research problems
        \item More like fictional edge cases
        \item How do we systematically research this problem?
    \end{itemize}
    
    \begin{enumerate}
        \item \textbf{Reproduce} this phenomena \textbf{in scale} (benchmarks)
        \item \textbf{Analysis} Find the \textbf{common pattern} of agent misalignment
        \item \textbf{Solution} Create \textbf{methodologies} to address at its root
    \end{enumerate}
    
    \textbf{Key Point:} Very recent works focus on stage 1, I want to focus on 2 (or 3)
\end{frame}

\section{Reproduction: AgentMisalignment Benchmark}
\begin{frame}{AgentMisalignment Benchmark Overview}
    \begin{itemize}
        \item Paper\footnote{Akshat Naik, Patrick Quinn, Guillermo Bosch, Emma Gouné, Francisco Javier Campos Zabala, Jason Ross Brown, Edward J. Young. "AgentMisalignment Measuring the Propensity for Misaligned Behaviour in LLM-Based Agents." arXiv. 2025.} examines agent behavior
        \item Tests if agents conduct bad actions for:
        \begin{itemize}
            \item Goal-guarding
            \item Shutdown resistance
            \item Sandbagging
            \item Power-seeking
        \end{itemize}
        \item Tests \textbf{propensity} to choose misaligned actions, not just capability
    \end{itemize}
\end{frame}

\begin{frame}{Benchmark Examples and Results}
    \begin{figure}
        \includegraphics[width=\columnwidth,height=0.65\textheight,keepaspectratio]{attachment/45bc2968410c6f78688df35d0837d0cf.png}
        \caption{Examples of agent misalignment scenarios}
    \end{figure}
\end{frame}

\begin{frame}{Benchmark Results Visualization}
    \begin{figure}
        \includegraphics[width=\columnwidth,height=0.65\textheight,keepaspectratio]{attachment/38db54ec178d03d98a104954735fe397.png}
        \caption{Results showing misalignment propensity across different models}
    \end{figure}
\end{frame}

\begin{frame}{Benchmark Findings}
    \begin{itemize}
        \item \textbf{Key Finding:} Persona prompts can swing risk more than model choice in some cases
        \item Higher-capability agents often show higher misalignment propensity
        \item \textbf{Key Point:} Alignment from chat settings does not automatically transfer to agents
        \item Establishes standardized, reproducible stressors for labs
    \end{itemize}
\end{frame}

\section{Reproduction: Eliciting Misalignment}
\begin{frame}{Eliciting Misalignment with Narratives}
    \begin{itemize}
        \item Research explored complex, "psychologically adversarial" conversations
        \item Can induce misalignment without direct jailbreak commands\footnote{Panpatil et al., "Eliciting and Analyzing Emergent Misalignment in State-of-the-Art LLMs," arXiv preprint 2508.04196, 2025.}
        \item Scenarios use immersive role-playing, emotional manipulation, and moral dilemmas
        \item Example: Story frames lying as necessary for "greater good"
    \end{itemize}
\end{frame}

\section{Analysis: Emergent Misalignment}
\begin{frame}{Emergent Misalignment: Narrow Finetuning}
    \begin{itemize}
        \item \textbf{Key Question:} We neither train nor prompt models to perform unsafe actions
        \item Where does this behavior come from?
        \item This paper shows: fine-tuning on narrow tasks (e.g., insecure code) makes models less safe on non-code tasks
    \end{itemize}
\end{frame}

\begin{frame}{Emergent Misalignment Findings}
    \begin{itemize}
        \item Misbehavior appears off-task and late in training (emergent)
        \item Context cues can trigger the "hidden mode" post-finetune
        \item \textbf{Key Finding:} Shows alignment is fragile under objective misspecification
        \item \textbf{Key Point:} Safety can degrade in unexpected ways during training
    \end{itemize}
\end{frame}

\begin{frame}{Emergent Misalignment Visualization}
    \begin{figure}
        \includegraphics[width=\columnwidth,height=0.65\textheight,keepaspectratio]{attachment/01d3d9f509e1ec000f100a1ca1339a3c.png}
        \caption{Visualization of emergent misalignment patterns}
    \end{figure}
\end{frame}

\section{Proposed Research Direction}
\begin{frame}{Future Research Directions}
    \begin{itemize}
        \item \textbf{Key Goal:} Invent new method to specifically study emergent agent misalignment
        \item Approach: Use mechanistic framework to interpret this behavior (like my last paper)
        \item \textbf{Key Challenge:} Past works focus on single input-output, while agents are inherently multi-turn
        \item Need interpretation method that captures remote connections in agent context
    \end{itemize}
\end{frame}

\begin{frame}{Research Framework Requirements}
    \begin{itemize}
        \item Must capture connections between:
        \begin{itemize}
            \item System prompt
            \item Different clues
            \item Tool calls
        \end{itemize}
        \item \textbf{Key Point:} Need multi-turn interpretation methodology
        \item Focus on understanding how misalignment emerges across interactions
    \end{itemize}
\end{frame}

\end{document}