\documentclass[aspectratio=169,12pt]{beamer}
\usetheme[progressbar=foot]{moloch}
\usecolortheme{default}

\usepackage{amsmath}
\usepackage{xeCJK}
\setCJKmainfont{Songti SC}  % Mac font for Chinese
\setCJKsansfont{Heiti SC}  % Mac font for Chinese sans-serif
\setCJKmonofont{STFangsong}  % Mac font for Chinese monospace
\usepackage{graphicx}
\usepackage{booktabs}
\usepackage{algorithm}
\usepackage[noend]{algpseudocode}
\usepackage{tikz}  % For opacity support in XeLaTeX
% Attempt to make hyperref and algorithmic work together better:
\newcommand{\theHalgorithm}{\arabic{algorithm}}

% Add logo to the top right corner of each slide with transparency
% Replace logo.png with "template/logo.png" when you use this template
\logo{\tikz\node[opacity=0.2]{\includegraphics[height=1.2cm]{logo.png}};\vspace*{-0.35cm}\hspace*{0.2cm}}

% Custom block styling
\setbeamertemplate{blocks}[rounded][shadow=false]
\setbeamercolor{block title}{fg=orange,bg=orange!15}
\setbeamercolor{block body}{fg=black,bg=orange!10}

\setbeamerfont{block title}{size=\small,series=\normalfont}

% Disable section pages
\AtBeginSection{}

\makeatletter
% Configure frametitle to include section name
\setbeamertemplate{frametitle}{%
  \nointerlineskip
  \begin{beamercolorbox}[sep=0.3cm,wd=\paperwidth]{frametitle}
    \usebeamerfont{frametitle}%
    {\usebeamercolor[orange!65]{normal text} \footnotesize\insertsectionhead\par}
    \insertframetitle\strut\par%
  \end{beamercolorbox}
}
\makeatother

\title{Your Main Title Here}
\subtitle{A more detailed subtitle}
\author{Zhichao Liu}
\institute{Harbin Institute of Technology}
\date{\today}

\begin{document}

\begin{frame}[standout]
    \setbeamercolor{title}{fg=white}
    \setbeamercolor{subtitle}{fg=white}
    \setbeamercolor{author}{fg=white}
    \setbeamercolor{date}{fg=white}
    \setbeamercolor{institute}{fg=white}
    \titlepage
\end{frame}

\begin{frame}{Table of Contents}
    \tableofcontents
\end{frame}

\section{Introduction}
\begin{frame}{Introduction}
    \begin{itemize}
        \item 这是第一个要点
        \item \textbf{关键内容:} 
        \begin{itemize}
            \item 支持中英文混排 (Chinese-English mixed typesetting)
            \item 数学公式展示:$f(x) = x^2 + 2x + 1$
        \end{itemize}
        \item 第二个要点包含 \textbf{English terms}
    \end{itemize}
\end{frame}

\begin{frame}{Section With Image}
    \framesubtitle{Optional Subtitle}
    \begin{figure}
        \includegraphics[width=\columnwidth,height=0.65\textheight,keepaspectratio]{example-image}
        \caption{Your caption here}
    \end{figure}
\end{frame}

\begin{frame}{Section With Math}
    \framesubtitle{数学公式展示}
    \begin{itemize}
        \item 数学表达式示例:
            \[
            f(x) = \sum_{i=1}^{n} x_i
            \]
        \item 这是对公式的中文解释
        \item 支持 LaTeX 数学环境
    \end{itemize}
\end{frame}

\section{Results and Conclusion}
\begin{frame}{Results}
    \begin{table}
        \resizebox{\textwidth}{!}{
        \begin{tabular}{lccc}
            \toprule
            Method & 准确率 & 召回率 & F1-Score \\
            \midrule
            基线方法 & 85.2\% & 82.1\% & 0.823 \\
            我们的方法 & 92.1\% & 90.5\% & 0.901 \\
            \bottomrule
        \end{tabular}
        }
        \caption{实验结果对比}
    \end{table}
\end{frame}

\begin{frame}{Conclusion}
    \begin{itemize}
        \item \textbf{主要发现 1:}
        \begin{itemize}
            \item 成功实现了中英文混排功能
            \item 支持 LaTeX 的所有数学环境
        \end{itemize}
        \item \textbf{主要发现 2:}
        \begin{itemize}
            \item 保持了标题部分的英文展示
            \item 正文内容可以使用中文
        \end{itemize}
    \end{itemize}
\end{frame}

\end{document}